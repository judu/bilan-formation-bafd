\documentclass[titlepage,11pt,a4paper]{article}
\usepackage{xunicode}
\usepackage{fontspec}
\usepackage[frenchb]{babel}
\usepackage[hidelinks]{hyperref}

\setmainfont[Ligatures=TeX]{Linux Libertine O}

\title{Bilan de formation BAFD}
\author{Julien Durillon}

\parskip 1mm

\begin{document}

\maketitle

%\tableofcontents

\section{Mon parcours d'animateur à directeur}

\subsection{Mon engagement dans le scoutisme}

Je suis arrivé dans le scoutisme en 1998 en tant que jeune aux Scouts de France puis aux Scouts et Guides
de France (identifiés dans ce bilan sous le sigle SGDF). Cet engagement dans le scoutisme
m'a guidé vers un rôle d'animateur à partir de 2009.

J'ai endossé ce rôle d'animateur en premier lieu pour rendre bénévolement au mouvement ce
qu'il m'avait apporté, en participant à l'éducation de jeunes guides et scouts. Puis les
formations BAFA que j'ai effectuées chez les SGDF, et notamment l'approfondissement, m'ont
permis de me familiariser avec le projet éducatif de l'association. Ce faisant, mon
engagement est passé d'un simple «\,rendre aux suivant ce que j'ai reçu\,» à une
appropriation du projet éducatif de l'association.

J'ai été animateur bénévole aux SGDF pendant cinq ans dont deux ans en tant que directeur
d'accueil de scoutisme. Pour ces deux dernières années, j'étais qualifié Directeur du Scoutisme Français.

Pendant ces années d'animation et de formation, j'ai pu découvrir le projet éducatif des
Scouts et Guides de France. J'ai pu apprendre à me positionner par rapport au projet
éducatif et à construire des projets pédagogiques répondant à ce dernier et aux besoins
des jeunes.

Pendant ces cinq camps, j'ai pu expérimenter plusieurs postes: responsable sanitaire,
responsable de l'intendance, trésorier et enfin directeur d'accueil de scoutisme.

\subsection{Une volonté de me former et de former}

Au terme de ces cinq années d'animation, j'ai pris un poste d'accompagnateur pédagogique.
Mon rôle était d'accompagner les maîtrises sur mon département à la réalisation
d'activités et à la construction de projets pendant l'année et pour les camps d'été.
Je me suis donc retrouvé dans le rôle de formateur. Et en tant que tel, j'avais le
devoir d'aller me former moi-même.

En parallèle de cette envie de me former pour être plus à l'aise dans mon rôle,
il existe un besoin de directeurs de formation BAFA au sein des SGDF. J'ai donc été appelé
à m'inscrire en cursus BAFD. Cette qualification me permettra d'organiser et diriger des
camps accompagnés et de diriger des formations agréées BAFA au sein du Scoutisme Français.

Un camp accompagné, chez les SGDF, est un accueil de scoutisme déclaré par l'échelon
territorial ou national. Il propose à des unités\footnote{L'ensemble des jeunes et des
responsables d'un accueil de scoutisme à l'année et en camp.} ne possédant pas
de directeur de vivre des camps scouts. L'équipe de direction accompagne les différents
camps installés sur le lieu du séjour.

\subsection{Mon projet de initial de formation}

Au début de mon cursus BAFD, en arrivant à la formation générale, j'avais déjà dirigé deux
accueils de scoutisme.

J'avais déjà expérimenté la fonction de directeur d'accueil collectif de mineur (ACM):
prendre en charge les tâches administratives; travailler avec une équipe d'animateurs;
construire les projets pédagogiques de chaque accueil. J'ai également eu à suivre des
stagaires BAFA pendant les accueils de scoutisme que j'ai dirigés avant de commencer ma
formation BAFD.

La session de formation générale m'a permis de prendre du recul sur mes pratiques, de
faire le point sur mon parcours de formation et d'évaluer mon positionnement par rapport
aux fonctions du directeur d'accueil collectif de mineurs.

\subsubsection{Mes atouts}

En début de parcours, mes points forts résidaient dans:

\begin{itemize}
   \item La construction et le suivi d'un projet pédagogique prenant en compte le
      projet éducatif de la structure ainsi que les besoins des jeunes;
   \item La prise de décision, la transmission de consigne, la répartition de la charge;
   \item La gestion de la vie quotidienne d'un accueil de scoutisme;
   \item La communication auprès des familles des jeunes, avant, pendant et après le
      séjour;
   \item La capacité à assurer la sécurité physique et affective des enfants et des
      jeunes.
\end{itemize}

En fin de parcours, je reconnais que j'étais à l'aise dans l'animation d'une équipe
de personnes que j'ai appris à connaître pendant l'année, en phase avec mes valeurs,
dans une relation de confiance construite sur plusieurs mois.

\subsubsection{Les axes d'amélioration}

Mon auto-évaluation de début de parcours a fait remonter les axes d'amélioration suivant:

\begin{itemize}
   \item La gestion administrative et financière de d'accueil;
   \item L'accompagnement d'un adulte en formation;
   \item La communication au sein de l'équipe d'animation;
   \item La mise en place de partenariats dans le cadre de l'accueil;
   \item Le suivi administratif de l'accueil au quotidien;
   \item La mise en place d'une démarche d'auto-évaluation des animateurs;
   \item La connaissance d'une autre association et d'un autre projet éducatif;
   \item Le recrutement d'un animateur.
\end{itemize}

J'ai donc construit mon parcours de formation pour répondre à ces axes d'amélioration.

\section{Chronologie de mon cursus}

Toutes les étapes de mon parcours se sont déroulées au sein des Scouts et Guides de
France, avec une nuance sur mon deuxième stage pratique.

Ma volonté d'effectuer cette formation dans cette association vient du fait qu'une
qualification BAFD me servira surtout à diriger des camps accompagnés et des formations
avec le Scoutisme Français.

Les Scouts et Guides de France font partie des 9 associations de scoutisme agréées par
l'état et par conséquent organisent des «\,accueils de scoutisme\,» tels que définis par
l'article 227-1 du code de l'action sociale et des familles. Le but des SGDF est d'éduquer
des enfants et des jeunes et les inviter à être des citoyens heureux, utiles et artisans
de paix. Pour ce faire, les SGDF proposent d'utiliser la méthode scoute, composée de 7
éléments:

\paragraph{La loi et la promesse} Les jeunes sont invités à construire leur règles de vie
en communauté et à s'engager à faire de leur mieux pour respecter ces règles.

\paragraph{La relation éducative} Les jeunes ne peuvent s'épanouir que dans un
environnement sain et dans une relation de confiance avec les adultes qui les encadrent.

\paragraph{Le cadre symbolique} La mise en place d'une culture et d'un langage communs
permet de structurer la vie en communauté et le chemin du jeune vers l'autonomie.

\paragraph{La vie dans la nature} Car la nature, la forêt est un formidable terrain de jeux
et d'expérimentations. Et c'est en vivant dans la nature qu'on apprend à la connaître et à
la respecter.

\paragraph{La vie en équipe} Les jeunes sont regroupés en équipe de 4 à 8 personnes pour
la vie quotidienne ainsi que la réalisation de projets.

\paragraph{La progression personnelle} Chaque jeune peut se constituer un parcours
de progression à son rythme. Ce parcours de progression invite le jeune à se dépasser et à
grandir.

\paragraph{L'éducation par l'action} Les jeunes sont invités à expérimenter la vie
quotidienne, à agir selon leurs convictions et à apprendre en faisant.

Pour se donner les moyens de faire vivre la méthode scoute aux enfants et aux jeunes, les
Scouts et Guides de France sont aussi organisme de formation BAFA et BAFD. Leurs
formations se basent sur la méthode scoute et contiennent en plus des objectifs du BAFA et
du BAFD des volets répondant aux objectifs de formation des responsables du mouvement.

\subsection{Session générale}

J'ai effectué ma session générale chez les SGDF en avril 2014, lors d'une formation
appelée \emph{Cham}. Outre la réalisation du cahier des charges de la formation générale BAFD,
le \emph{Cham} propose aux stagiaires de creuser les origines de la méthode scoute et de
questionner la pédagogie mise en place par les SGDF. Cette session propose deux volets: le
développement du scoutisme et l'accompagnement des bénévoles.

J'ai choisi le parcours «\,accompagnement\,». Cela m'a permis de prendre du recul sur mes
pratiques et d'obtenir des connaissances théoriques utiles pour ma mission
d'accompagnateur et de directeur d'ACM.

\subsection{Premier stage pratique}

Mon premier stage pratique s'est déroulé sur deux accueils de scoutisme:

\paragraph{Le premier accueil} était un camp comportant une itinérance en bateau d'une semaine suivi d'une
deuxième semaine fixe. Pour des raisons techniques, la direction sur place de l'itinérance
a été confiée à un adjoint, titulaire du «\,Certificat d'Aptitude aux fonctions de Directeur
du Scoutisme Français\,», ainsi que de la qualification «\,Chef de Flotille\,» défini par
l'arrêté du 26 juin
2008\footnote{https://www.legifrance.gouv.fr/eli/arrete/2008/6/26/SJSF0815795A/jo/texte}.
J'étais disponible en soutien pour la communication et les questions administratives de
l'accueil.

J'ai ensuite pris la direction sur place pour la deuxième semaine. Un de mes rôle sur ce
stage était d'accompagner et d'évaluer un animateur ayant effectué une session
d'approfondissement BAFA au retour mitigé, notamment sur ses capacités à exercer un rôle
de directeur d'Accueil de Scoutisme.

J'avais une équipe de trois animateurs et une animatrice pour 27 jeunes de 14 à 17 ans. Les animateurs
étaient tous qualifiés Animateurs du Scoutisme Français, aucun n'était en cursus BAFA.

\paragraph{Le deuxième accueil} s'est déroulé le même été: 23 jeunes de 8 à 11 ans
encadrés par deux animateurs et une animatrice. Un des animateurs avait la qualification
«\,Animateur Stagiaire du Scoutisme Français\,», l'autre était non qualifié. Les deux
avaient pratiqué le rôle d'animateurs pendant l'année avec ce groupe de jeunes.

Enfin, l'animatrice était qualifiée «\,Directrice du Scoutisme Français\,». Elle m'a
recruté sur ce camp suite à mon accompagnement pendant l'année. Mon but sur ce camp était
de gérer au mieux la partie administrative et financière, ainsi que la gestion de l'équipe
d'animation. J'ai délégué à cette animatrice le suivi du projet pédagogique et du projet
d'activité pour lesquels j'avais accompagné la maîtrise\footnote{Voir lexique} pendant la
préparation du camp.

Ce camp était plus confortable que le précédent, et j'ai pu en profiter pour essayer
de m'acquitter exemplairement de ma tâche administrative et financière.

\subsection{Session de perfectionnement}

J'ai effectué ma session de perfectionnement chez les SGDF, au cours d'un stage appelé
\emph{STAF}. Ce stage propose deux volets: formation (et direction de formation) et
management.

J'y ai choisi le parcours «\,formation\,». Cela m'a apporté beaucoup d'outils et de
techniques pour être à l'aise et formation et en direction de formation.

\subsection{Deuxième stage pratique}

Mon deuxième stage pratique est né de la volonté de réaliser un projet plus ambitieux, de
découvrir d'autres associations et projets éducatifs et d'expérimenter la mise en place de
partenariats. Cette dernière partie correspondait à un manque dans les fonctions du
directeur que j'avais expérimentées.

L'accueil a été déclaré par les SGDF, mais a regroupé trois associations du Scoutisme
Français: les Éclaireuses et Éclaireurs de France (EEDF), les Éclaireuses et Éclaireurs
Unionistes de France (EEUDF) et les Scouts et Guides de France (SGDF).

L'idée de ce camp était de réunir plusieurs associations du Scoutisme Français sur un camp
pour vivre un moment d'échange et de découverte d'autres façons de faire. J'ai contacté
en novembre 2016 les différentes associations présentes à Nantes. Sur les quatre associations
présentes à Nantes, les trois citées plus haut m'ont répondu.


La structure de l'accueil était un «\,camp accompagné\,»: Je gérais une équipe de
direction et chacun des camps présents (quatre la première semaine, trois la deuxième)
était indépendant des autres pour la vie quotidienne et la plupart des activités. Nous
avions néanmoins préparé des temps en commun, que je décrirai plus bas dans ce bilan.

L'accueil était dirigé par moi, secondé par une adjointe SGDF (Coline) en premier stage pratique BAFD
ainsi que d'une responsable EEDF (Maguelone) en fin de parcours BAFD et d'un responsable
EEUDF (Nicolas) en stage pratique «\,Directeur du Scoutisme Français\,».
Ce deux derniers adjoints avaient pour rôle de gérer leurs sous-camps respectifs.
Pour les deux sous-camps restant, Coline et moi nous sommes partagé le suivi des activités, du
projet pédagogique et des animateurs stagiaires.

Chaque sous-camp avait sa propre équipe d'animateurs et ses jeunes, pour un total de 23
animateurs et 100 jeunes simultanéement sur l'accueil complet.

Pour préparer ce camp, nous avons travaillé en équipe de direction sur un projet
pédagogique global, ainsi que des temps et un imaginaire commun. Pour ce faire, nous avons
commencé par lire ensemble les projets éducatifs des différentes associations pour relever
les points en commun et les différences entre ces projets et nous imprégner de leurs
valeurs.

Nous avons ainsi relevé, outre les différences religieuses à la base de ces différentes
associations (Catholiques, Protestants, Laïcs), certaines différences. Des considérations sur
le port ou non d'une tenue commune, la volonté chez les SGDF d'une éducation
inter-culturelle, inter-générationnelle, inter-religieuse, la présence forte de la
démocratie dans le groupe chez les EEDF, l'engagement personnel dans la vie locale et
sociale prôné par les EEUDF.

Nous avons néanmoins décidé sur ce camp de se concentrer sur les points communs entre les
mouvements, pour lutter contre les préjugés entre les mouvements de scoutisme, qui ne se
côtoient pas si souvent. Le projet pédagogique de ce camp se trouve dans les annexes.


\section{Expérimenter et approfondir les fonctions du directeur}

Je déclinerai ici mon parcours de formation sur chacune des fonctions du directeur
d'accueil collectif de mineurs, en les ordonnant chronologiquement: j'ai choisi mes stages
pratiques de manière à me concentrer sur certaines des fonctions du directeur d'ACM à
chaque étape.

\subsection{Diriger des personnels}

Comme énoncé plus tôt, j'avais identifié en début de parcours des points forts sur la
gestion de personnels, tout en détectant un manque de pratique dans le suivi de stagiaire
en formation et la communication dans l'équipe d'animation.

\subsubsection{Session de formation générale}

Lors de la session de formation générale, j'ai pu prendre du recul sur mes pratiques
précédentes et apprendre des techniques de gestion d'équipe et d'accompagnement
d'adultes.

J'ai également pu échanger avec mes formateurs ainsi que mes pairs sur mes expériences
passées. Cela m'a permis de trouver pistes pour mieux gérer des situations dans lesquelles
je ne m'étais pas senti à l'aise.

\subsubsection{Premier stage pratique}

Mon premier stage pratique a été effectué dans deux accueils, le même été. J'ai dirigé le
premier accueil dans le but de suivre un stagiaire qui avait été problématique pendant
l'année et qui avait fait un stage d'approfondissement "Accueil de Scoutisme" à
l'évaluation mitigée. Il n'avait pas été qualifié comme Directeur du Scoutisme Français et
un de mes rôles dans ce stage était de l'accompagner et l'évaluer pour que l'association puisse (ou
non) le qualifier par la suite.

Le but de ce premier stage était de me permettre d'expérimenter l'accompagnement d'un
adulte en formation et de favoriser la communication au sein de l'équipe d'animation. Il y
avait une équipe de 4 animateurs en plus de moi.

Ce stage m'a fait prendre conscience que ce que j'avais identifié comme un point fort
(gestion d'une équipe d'animateur) devait plus à un facteur externe. En effet, les équipes
d'animateurs que j'avais dirigées auparavant s'étaient construites et rodées avec moi pendant
une année voire plus. L'ambiance et la confiance y étaient naturelles. À l'inverse, sur ce
stage, je me suis retrouvé à diriger une équipe constituée tardivement, de personnes que
je n'avais pas cotoyées en animation et où une certaine tension entre ses membres
existait.

Pendant ce stage, j'ai dû faire face à différentes situations de conflit entre des
animateurs et quelques fois entre des animateurs et des jeunes. Ces situations de conflit
ont pu se résoudre, mais partant avec un \textit{a priori} sur le responsable que je
devais accompagner, il m'a été difficile de rester impartial, devant plus faire preuve
d'autorité pour en résoudre certaines.

\subsubsection{Deuxième stage pratique}

J'ai vécu mon rôle de directeur de façon très différente sur ce stage. En effet,
je n'étais plus en position d'animateur-directeur sur ce camp, mais uniquement en
direction, ayant assez peu de contacts avec les jeunes par rapport aux animateurs.

J'ai donc mis en place des réunions régulières avec des responsables des sous-camps.
Quotidiennes au début, plus espacées par la suite pour laisser les responsables prendre leur autonomie.
Avec mon adjointe, nous avons aussi préparé des temps d'échange pour tous les responsable et les faire
participer à la conception et à la préparation des activités communes.

Une partie de notre rôle de directeurs a consisté à aider les responsables les moins à l'aise à
la mise en place de processus favorisant le respect des normes d'hygiène et de la sécurité.

J'ai également pu accompagner l'auto évaluation de Coline pour son cursus BAFD ainsi que
de Nicolas sur son cursus de formation à la direction d'accueil de scoutisme. J'ai
également dû aider une des maîtrises qui avait des souci avec une animatrice qui ne se
sentait pas à sa place. Nous avons pu discuter ensemble et trouver un moyen de l'aider à
prendre son rôle d'animatrice responsable face aux jeunes.

\subsection{Conduire un projet pédagogique en référence au projet éducatif}

\subsubsection{Points forts}

Comme indiqué au début de ce document, j'étais plutôt à l'aise au début de mon cursus BAFD
avec le fait de construire un projet pédagogique tenant compte des besoins des jeunes et des
envies de l'équipe d'animation en cohérence avec le projet éducatif de l'association
organisatrice.

\subsubsection{Premier stage pratique}


Sur mon premier stage pratique, j'ai accompagné les maîtrises lors de l'évaluation de
leurs projets d'année et la rédaction de leur projet pédagogique et projet d'activité de
camp. Les animateurs étaient ainsi pleinement possesseurs de leur projet pédagogique.

J'ai également procédé avec les animateurs à l'évaluation pendant et après le camp du
projet pédagogique.

\subsubsection{Deuxième stage pratique}

En préparant l'accueil de mon deuxième stage pratique, j'ai étudié avec l'équipe de
direction les projets éducatifs de chaque association, en demandant notamment aux
responsables de chaque association de nous présenter leur projet éducatif respectif
et de nous souligner les points forts à leurs yeux.

Dans chacun des mouvements de scoutisme, une unité partant en camp doit écrire un
projet pédagogique qui est validé par son échelon territorial. Sur mon accueil, il y avait
donc un projet pédagogique commun décrivant les envies de l'équipe de direction pour le
projet commun et un projet pédagogique propre à chaque unité, représentant les besoins et
envie des maîtrises par rapport à la vie de leur unité pendant l'année et pour le camp.

En tant que directeur, j'ai aussi lu les projets pédagogiques de chaque sous-camp pour les
accompagner dans leur mise en place. En tant qu'accompagnateur pédagogique à Nantes, j'ai
également accompagné les maîtrises SGDf les moins formées à la rédaction du projet
pédadogique et du projet d'activité de leur sous-camp.

\subsection{Assurer la gestion de l'accueil}

\subsubsection{Le point en début de parcours}

Pendant mes deux années de direction, j'ai construit ma capacité à gérer un accueil de
scoutisme: la vie quotidienne, le budget, l'administratif, etc.

Néanmoins, j'ai toujours été soutenu par les secrétaires et responsables de mon groupe
local. J'avais plusieurs fois fait appel à eux pour gérer des entrées/sorties et une
partie du suivi administratif.

\subsubsection{Premier stage pratique}

Pendant le deuxième camp de ce stage pratique, je me suis concentré sur la gestion
administrative de l'accueil: le suivi des jeunes, le respect des règles de sécurité et
d'hygiène, le contrôle du budget quotidien.

Nous étions sur une base appartenant aux SGDF et partagions des réfrigérateurs avec
d'autres camps. J'ai participé au suivi de leur température, collecté des échantillons de
chaque repas. Il y a eu deux journées très chaudes pendant le camp, et le réfrigérateur
étant en extérieur, la température est montée à un niveau ne permettant pas de garantir la
conservation de certains aliments (viande ou produits laitiers par exemple). Avec l'intendant, j'ai dû
changer des menus pour éviter d'acheter de la viande et remplacer certains produits frais
par d'autres pour les repas suivant.

\subsubsection{Deuxième stage pratique}

% Présentation du budget pédagogique.


\subsection{Développer les partenariats et la communication}
\subsubsection{Points forts}
\subsubsection{Axe d'amélioration: deuxième stage pratique}

\subsection{Situer son engagement dans le contexte social, culturel et éducatif}
\subsubsection{Points forts}
\subsubsection{Axe d'amélioration: deuxième stage pratique}

\section{Faire le point en fin de parcours}

\end{document}
